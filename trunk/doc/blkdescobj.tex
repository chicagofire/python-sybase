\subsection{CS_BLKDESC Objects}

Calling the \method{blk_alloc()} method of a \class{CS_CONNECTION}
object will create a \class{CS_BLKDESC} object.  When the
\class{CS_BLKDESC} object is deallocated the Sybase
\function{blk_drop()} function will be called for the command.

\class{CS_BLKDESC} objects have the following interface:

\begin{methoddesc}[CS_BLKDESC]{blk_bind}{num, databuf}
Calls the Sybase \function{blk_bind()} function passing the \var{num}
and \var{databuf} arguments.  It returns the Sybase result code.  The
Sybase-CT \function{blk_bind()} function is called like this:

\begin{verbatim}
status = blk_bind(blk, num, &datafmt, buffer->buff, buffer->copied, buffer->indicator);
\end{verbatim}
\end{methoddesc}

\begin{methoddesc}[CS_BLKDESC]{blk_describe}{num}
Calls the Sybase \function{blk_describe()} function passing the
\var{num} argument and returns a tuple containing the Sybase result
code and a \class{CS_DATAFMT} object which describes the column
identified by \var{num}.  \code{None} is returned as the
\class{CS_DATAFMT} object when the result code is not
\code{CS_SUCCEED}.

The Sybase \function{blk_describe()} function is called like this:

\begin{verbatim}
status = blk_describe(blk, num, &datafmt);
\end{verbatim}
\end{methoddesc}

\begin{methoddesc}[CS_BLKDESC]{blk_done}{type}
Calls the Sybase \function{blk_done()} function passing the
\var{type} argument and returns a tuple containing the Sybase result
code and the number of rows copied in the current batch.

The Sybase \function{blk_done()} function is called like this:

\begin{verbatim}
status = blk_done(blk, type, &num_rows);
\end{verbatim}
\end{methoddesc}

\begin{methoddesc}[CS_BLKDESC]{blk_drop}{}
Calls the Sybase \function{blk_drop()} function and returns the Sybase
result code.

The Sybase \function{blk_drop()} function is called like this:

\begin{verbatim}
status = blk_drop(blk);
\end{verbatim}

This method will be automatically called when the \class{CS_BLKDESC}
object is deleted.  Applications do not need to call the method.
\end{methoddesc}

\begin{methoddesc}[CS_BLKDESC]{blk_init}{direction, table}
Calls the Sybase \function{blk_init()} function and returns the Sybase
result code.

The Sybase \function{blk_init()} function is called like this:

\begin{verbatim}
status = blk_init(blk, direction, table, CS_NULLTERM);
\end{verbatim}
\end{methoddesc}

\begin{methoddesc}[CS_BLKDESC]{blk_props}{action, property \optional{, value}}
\end{methoddesc}

\begin{methoddesc}[CS_BLKDESC]{blk_rowxfer}{}
Calls the Sybase \function{blk_rowxfer()} function and returns the
Sybase result code.

The Sybase \function{blk_rowxfer()} function is called like this:

\begin{verbatim}
status = blk_rowxfer(blk);
\end{verbatim}
\end{methoddesc}

\begin{methoddesc}[CS_BLKDESC]{blk_rowxfer_mult}{\optional{row_count}}
Calls the Sybase \function{blk_rowxfer_mult()} function and returns a
tuple containing the Sybase result code and the number of rows
transferred.

The Sybase \function{blk_rowxfer_mult()} function is called like this:

\begin{verbatim}
status = blk_rowxfer_mult(blk, &row_count);
\end{verbatim}
\end{methoddesc}

\begin{methoddesc}[CS_BLKDESC]{blk_textxfer}{\optional{string}}
\end{methoddesc}

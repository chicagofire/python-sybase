\subsection{CS_DATAFMT Objects}

\class{CS_DATAFMT} objects are a very thing wrapper around the
Sybase \code{CS_DATAFMT} structure.  They have the following
attributes:

\begin{tabular}{l|l}
attribute & type \\
\hline
\code{name}      & \code{string} \\
\code{datatype}  & \code{int} \\
\code{format}    & \code{int} \\
\code{maxlength} & \code{int} \\
\code{scale}     & \code{int} \\
\code{precision} & \code{int} \\
\code{status}    & \code{int} \\
\code{count}     & \code{int} \\
\code{usertype}  & \code{int} \\
\code{strip}     & \code{int} \\
\end{tabular}

The \code{strip} attribute is an extension of the Sybase
\code{CS_DATAFMT} structure.  Please refer to the \class{DataBuf}
documentation.

\class{CS_DATAFMT} structures are mostly used to create
\class{DataBuf} objects for sending data to and receiving data from
the server.

A \class{CS_DATAFMT} object created via the \function{CS_DATAFMT()}
constructor will have the following values:

\begin{tabular}{l|l}
attribute & value \\
\hline
\code{name}      & \code{'$\backslash$0'} \\
\code{datatype}  & \code{CS_CHAR_TYPE} \\
\code{format}    & \code{CS_FMT_NULLTERM} \\
\code{maxlength} & \code{1} \\
\code{scale}     & \code{0} \\
\code{precision} & \code{0} \\
\code{status}    & \code{0} \\
\code{count}     & \code{0} \\
\code{usertype}  & \code{0} \\
\code{strip}     & \code{0} \\
\end{tabular}

You will almost certainly need to provide new values for some of the
attributes before you use the object.

A \class{CS_DATAFMT} object created as a return value from the
\function{ct_bind()} function will be ready to use for creating a
\class{DataBuf} object.

\chapter{DB-API 2.0 Interface}

\localmoduletable

\section{\module{Sybase} --- Provide interface to Sybase relational database}

\declaremodule{standard}{Sybase}

%\modulesynopsis{DB-API 2.0 interface to Sybase.}

The \module{Sybase} module contains the following:

\begin{datadesc}{__version__}
A string which specifies the version of the Sybase module.
\end{datadesc}

\begin{datadesc}{apilevel}
Specifies the level of DB-API compliance.  Currently set to
\texttt{'2.0'}.
\end{datadesc}

\begin{datadesc}{threadsafety}
Specifies the DB-API threadsafety.  The \module{Sybase} module allows
threads to share the module, connections and cursors.
\end{datadesc}

\begin{datadesc}{paramstyle}
Specifies the DB-API parameter style.  The \module{Sybase} module uses
question marks as parameter place holders.
\end{datadesc}

\begin{classdesc}{Cursor}{owner}
Return a new instance of the \class{Cursor} class which implements the
DB-API 2.0 cursor functionality.

The \var{owner} argument must be an instance of the \class{Connection}
class.
\end{classdesc}

\begin{classdesc}{Bulkcopy}{owner, table, direction}
Return a new instance of the \class{Bulkcopy} class.

The \var{owner} argument must be an instance of the \class{Connection}
class.  A bulk copy context will be established for the table named in
the \var{table} argument, the bulkcopy direction must be either
\code{CS_BLK_IN} or \code{CS_BLK_OUT} as defined in the
\module{_sybase} module.

This is an extension of the DB-API 2.0 specification.
\end{classdesc}

\begin{classdesc}{Connection}{dsn, user, passwd \optional{, \ldots}}
Return a new instance of the \class{Connection} class which implements
the DB-API 2.0 connection functionality.

The \var{dsn} argument identifies the Sybase server, \var{user} and
\var{passwd} are the Sybase username and password respectively.

The optional argument are the same as those supported by the
\function{connect()} function.
\end{classdesc}

\begin{funcdesc}{connect}{dsn, user, passwd \optional{, \ldots}}
Implements the DB-API 2.0 \function{connect()} function.

Creates a new \class{Connection} object passing the function arguments
to the \class{Connection} constructor.  The
optional arguments and their effect are:

\begin{description}
\item{\var{database}}

Specifies the database to use - has the same effect as the following
SQL.

\begin{verbatim}
use database
\end{verbatim}

\item{\var{strip}}

If non-zero then all \code{char} columns will be right stripped of
whitespace.

\item{\var{auto_commit}}

Controls Sybase chained transaction mode.  When non-zero, chained
transaction mode is turned off.  From the Sybase SQL manual:

\begin{quote}
If you set chained transaction mode, Adaptive Server implicitly
invokes a begin transaction before the following statements: delete,
insert, open, fetch, select, and update. You must still explicitly
close the transaction with a commit.
\end{quote}

\item{\var{bulkcopy}}

Must be non-zero if you are going to perform bulkcopy operations on
the connection.

\item{\var{delay_connect}}

If non-zero the returned \class{Connection} object will be initialised
but not connected.  This allows you to set additional options on the
connection before completing the connection to the server.  Call the
\method{connect()} method to complete the connection.

\begin{verbatim}
db = Sybase.connect('SYBASE', 'sa', '', delay_connect = 1)
db.set_property(Sybase.CS_HOSTNAME, 'secret')
db.connect()
\end{verbatim}
\end{description}
\end{funcdesc}

\subsection{Connection Objects}

Implements the DB-API 2.0 \class{Connection} class.

\class{Connection} objects have the following interface:

\begin{methoddesc}[Connection]{close}{}
Implements the DB-API 2.0 connection \method{close()} method.

Forces the database connection to be closed immediately.  Any
operation on the connection (including cursors) after calling this
method will raise an exception.

This method is called by the \method{__del__()} method.
\end{methoddesc}

\begin{methoddesc}[Connection]{commit}{\optional{name \code{= None}}}
Implements the DB-API 2.0 connection \method{commit()} method.

Calling this method commits any pending transaction to the database.
By default Sybase transaction chaining is enabled.  If you pass
\code{auto_commit = 1} to the \function{connect()} function when
creating this \class{Connection} object then chained transaction mode
will be turned off.

From the Sybase manual:
\begin{quote}
If you set chained transaction mode, Adaptive Server implicitly
invokes a begin transaction before the following statements: delete,
insert, open, fetch, select, and update. You must still explicitly
close the transaction with a commit.
\end{quote}

The optional \var{name} argument is an extension of the DB-API 2.0
specification.  Sybase allows you to nest and name transactions.
\end{methoddesc}

\begin{methoddesc}[Connection]{rollback}{\optional{name \code{= None}}}
\end{methoddesc}

\begin{methoddesc}[Connection]{cursor}{}
\end{methoddesc}

\begin{methoddesc}[Connection]{begin}{\optional{name \code{= None}}}
This is an extension of the DB-API 2.0 specification.
\end{methoddesc}

\begin{methoddesc}[Connection]{get_property}{prop}
This is an extension of the DB-API 2.0 specification.
\end{methoddesc}

\begin{methoddesc}[Connection]{set_property}{prop, value}
This is an extension of the DB-API 2.0 specification.
\end{methoddesc}

\begin{methoddesc}[Connection]{execute}{sql}
This is an extension of the DB-API 2.0 specification.
\end{methoddesc}

\begin{methoddesc}[Connection]{bulkcopy}{table \optional{, copy_out\code{ = 0}}}
This is an extension of the DB-API 2.0 specification.
\end{methoddesc}

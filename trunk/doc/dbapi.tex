\section{\module{Sybase} --- Provides interface to Sybase relational database}

\declaremodule{standard}{Sybase}

%\modulesynopsis{DB-API 2.0 interface to Sybase.}

The \module{Sybase} module contains the following:

\begin{datadesc}{__version__}
A string which specifies the version of the Sybase module.
\end{datadesc}

\begin{datadesc}{use_datetime}
When you import the \module{Sybase} module it will try to import the
\module{mxDateTime} module.  If the \module{mxDateTime} module is
successfully imported then this variable will be set to \code{1}.  All
\code{datetime} columns will then be returned as
\class{mxDateTime.DateTime} objects.

If you do not wish to use \class{mxDateTime.DateTime} objects, set
this to \code{0} immediately after importing the \module{Sybase}
module.  All \code{datetime} columns will then be returned as
\class{sybasect.DateTime} objects.
\end{datadesc}

\begin{datadesc}{apilevel}
Specifies the level of DB-API compliance.  Currently set to
\texttt{'2.0'}.
\end{datadesc}

\begin{datadesc}{threadsafety}
Specifies the DB-API threadsafety.  The \module{Sybase} module allows
threads to share the module, connections and cursors.
\end{datadesc}

\begin{datadesc}{paramstyle}
Specifies the DB-API parameter style.  The \module{Sybase} module uses
question marks as parameter place holders.
\end{datadesc}

\begin{classdesc}{Cursor}{owner}
Return a new instance of the \class{Cursor} class which implements the
DB-API 2.0 cursor functionality.

\class{Cursor} objects are usually created via the
\method{cursor()} method of the \class{Connection} object.

The \var{owner} argument must be an instance of the \class{Connection}
class.
\end{classdesc}

\begin{classdesc}{Bulkcopy}{owner, table, direction}
Return a new instance of the \class{Bulkcopy} class.

The \var{owner} argument must be an instance of the \class{Connection}
class.  A bulk copy context will be established for the table named in
the \var{table} argument, the bulkcopy direction must be either
\code{CS_BLK_IN} or \code{CS_BLK_OUT} as defined in the
\module{Sybase} module.

The \var{owner} argument must be an instance of the \class{Connection}
class.

\class{Bulkcopy} objects are usually created via the
\method{bulkcopy()} method of the \class{Connection} object.

This is an extension of the DB-API 2.0 specification.
\end{classdesc}

\begin{classdesc}{Connection}{dsn, user, passwd \optional{, \ldots}}
Return a new instance of the \class{Connection} class which implements
the DB-API 2.0 connection functionality.

The \var{dsn} argument identifies the Sybase server, \var{user} and
\var{passwd} are the Sybase username and password respectively.

The optional argument are the same as those supported by the
\function{connect()} function.
\end{classdesc}

\begin{funcdesc}{connect}{dsn, user, passwd \optional{, \ldots}}
Implements the DB-API 2.0 \function{connect()} function.

Creates a new \class{Connection} object passing the function arguments
to the \class{Connection} constructor.  The
optional arguments and their effect are:

\begin{description}
\item{\var{database}}

Specifies the database to use - has the same effect as the following
SQL.

\begin{verbatim}
use database
\end{verbatim}

\item{\var{strip}}

If non-zero then all \code{char} columns will be right stripped of
whitespace.

\item{\var{auto_commit}}

Controls Sybase chained transaction mode.  When non-zero, chained
transaction mode is turned off.  From the Sybase SQL manual:

\begin{quote}
If you set chained transaction mode, Adaptive Server implicitly
invokes a begin transaction before the following statements: delete,
insert, open, fetch, select, and update. You must still explicitly
close the transaction with a commit.
\end{quote}

\item{\var{bulkcopy}}

Must be non-zero if you are going to perform bulkcopy operations on
the connection.

\item{\var{delay_connect}}

If non-zero the returned \class{Connection} object will be initialised
but not connected.  This allows you to set additional options on the
connection before completing the connection to the server.  Call the
\method{connect()} method to complete the connection.

\begin{verbatim}
db = Sybase.connect('SYBASE', 'sa', '', delay_connect = 1)
db.set_property(Sybase.CS_HOSTNAME, 'secret')
db.connect()
\end{verbatim}
\end{description}
\end{funcdesc}

The \module{Sybase} module imports the low level \module{sybasect}
extension module via

\begin{verbatim}
from sybasect import *}
\end{verbatim}

which means that the \module{Sybase} module inherits all of the
objects defined in that module.

Some of the functions will be useful in your programs.

\begin{funcdesc}{datetime}{str \optional{, type \code{= Sybase.CS_DATETIME_TYPE}}}
Creates a new instance of the \class{DateTime} class.  This is used to
handle native Sybase \code{datetime} and \code{smalldatetime} values.

The string passed in the \var{str} argument is converted to a datetime
value of the type specifed in the optional \var{type} argument.

\code{Sybase.CS_DATETIME_TYPE} represents the \code{datetime} Sybase
type and \code{Sybase.CS_DATETIME4_TYPE} represents
\code{smalldatetime}.

The \class{DateTime} class is described in the \module{sybasect}
module.
\end{funcdesc}

\begin{funcdesc}{money}{num}
Creates a new instance of the \class{Money} class.  This is
used to handle native Sybase \code{money} values.

The value passed in the \var{num} argument is converted to a native
Sybase money value.

The \class{DateTime} class is described in the \module{sybasect}
module.
\end{funcdesc}

\begin{funcdesc}{numeric}{num, \optional{precision \code{= -1} \optional{, scale \code{= -1}}}}
Creates a new instance of the \class{Numeric} class.  This is used to
handle native Sybase \code{numeric} and \code{decimal} values.

Converts the value passed in the \var{num} argument to a native Sybase
numeric value.  The \var{precision} and \var{scale} arguments control
the precision and scale of the returned value.

The \class{Numeric} class is described in the \module{sybasect}
module.
\end{funcdesc}

\subsection{Connection Objects}

Implements the DB-API 2.0 \class{Connection} class.

\class{Connection} objects have the following interface:

\begin{methoddesc}[Connection]{close}{}
Implements the DB-API 2.0 connection \method{close()} method.

Forces the database connection to be closed immediately.  Any
operation on the connection (including cursors) after calling this
method will raise an exception.

This method is called by the \method{__del__()} method.
\end{methoddesc}

\begin{methoddesc}[Connection]{commit}{\optional{name \code{= None}}}
Implements the DB-API 2.0 connection \method{commit()} method.

Calling this method commits any pending transaction to the database.
By default Sybase transaction chaining is enabled.  If you pass
\code{auto_commit = 1} to the \function{connect()} function when
creating this \class{Connection} object then chained transaction mode
will be turned off.

From the Sybase manual:
\begin{quote}
If you set chained transaction mode, Adaptive Server implicitly
invokes a begin transaction before the following statements: delete,
insert, open, fetch, select, and update. You must still explicitly
close the transaction with a commit.
\end{quote}

The optional \var{name} argument is an extension of the DB-API 2.0
specification.  It allows you to make use of the Sybase ability to
nest and name transactions.
\end{methoddesc}

\begin{methoddesc}[Connection]{rollback}{\optional{name \code{= None}}}
Implements the DB-API 2.0 connection \method{rollback()} method.

Tells Sybase to roll back to the start of any pending transaction.
Closing a connection without committing the changes first will cause
an implicit rollback to be performed.

The optional \var{name} argument is an extension of the DB-API 2.0
specification.  It allows you to make use of the Sybase ability to
nest and name transactions.
\end{methoddesc}

\begin{methoddesc}[Connection]{cursor}{}
Implements the DB-API 2.0 connection \method{cursor()} method.

Returns a new \class{Cursor} object using the connection.
\end{methoddesc}

\begin{methoddesc}[Connection]{begin}{\optional{name \code{= None}}}
This is an extension of the DB-API 2.0 specification.

If you have turned off chained transaction mode via the
\var{auto_commit} argument to the connection constructor then you can
use this method to begin a transaction.  It also allows you to use the
nested transaction support in Sybase.
\end{methoddesc}

\begin{methoddesc}[Connection]{connect}{}
This is an extension of the DB-API 2.0 specification.

If you pass a TRUE value to the \var{delay_connect} argument to the
\function{connect()} function then you must call this method to
complete the server connection process.  This is useful if you wish to
set connection properties which cannot be set after you have connected
to the server.
\end{methoddesc}

\begin{methoddesc}[Connection]{get_property}{prop}
This is an extension of the DB-API 2.0 specification.

Use this function to retrieve properties of the connection to the
server.

\begin{verbatim}
import Sybase
db = Sybase.connect('SYBASE', 'sa', '', 'pubs2')
print db.get_property(Sybase.CS_TDS_VERSION)
\end{verbatim}
\end{methoddesc}

\begin{methoddesc}[Connection]{set_property}{prop, value}
This is an extension of the DB-API 2.0 specification.

Use this function to set properties of the connection to the server.
\end{methoddesc}

\begin{methoddesc}[Connection]{execute}{sql}
This is an extension of the DB-API 2.0 specification.

This method executes the SQL passed in the \var{sql} argument via the
\function{ct_command(CS_LANG_CMD, ...)} Sybase function.  This is what
programs such as \program{sqsh} use to send SQL to the server.  The
return value is a list of logical results.  Each logical result is a
list of row tuples.

The disadvantage to using this method is that you are not able to bind
binary parameters to the command, you must format the parameters as
part of the SQL string in the \var{sql} argument.
\end{methoddesc}

\begin{methoddesc}[Connection]{bulkcopy}{table, direction}
This is an extension of the DB-API 2.0 specification.

Returns a new \class{Bulkcopy} object using the connection.  The
\var{table} argument identifies the table which you wish to perform
bulkcopy operations upon.

The \var{direction} argument controls the direction of the bulkcopy
operation.  Specify \code{Sybase.CS_BLK_IN} to copy data in, or
\code{Sybase.CS_BLK_OUT} to copy data out.
\end{methoddesc}

\subsection{Cursor Objects}

Implements the DB-API 2.0 \class{Cursor} class.

\class{Cursor} objects have the following interface:

\begin{memberdesc}[Cursor]{description}
The DB-API 2.0 cursor \member{description} member.

A list of 7-item tuples.  Each of the tuples describes one result
column: \code{(name, type_code, display_size, internal_size,
precision, scale, null_ok)}.  This attribute will be \code{None} for
operations which do not return rows or if the cursor has not had an
operation invoked via \method{execute()} or \method{executemany()}.

The \code{type_code} can be interpreted by comparing it to the Type
Objects specified in the section below.
\end{memberdesc}

\begin{memberdesc}[Cursor]{rowcount}
The DB-API 2.0 cursor \member{rowcount} member.

This attribute reports the number of rows that the last
\method{execute()} or \method{executemany()} method produced or
affected.

The attribute is \code{-1} if no \method{execute()} or
\method{executemany()} has been performed on the cursor.
\end{memberdesc}

\begin{methoddesc}[Cursor]{callproc}{procname \optional{, parameters}}
Implements the DB-API 2.0 cursor \method{callproc()} method.

Calls the stored database procedure named in the \var{procname}
argument. The optional \var{parameters} argument must be a sequence
which contains one entry for each argument that the procedure expects.

The result of the call is returned as a modified copy of the input
sequence.  Input parameters are left untouched, output and
input/output parameters replaced with possibly new values.

The procedure may also provide a result set as output. This can be
retrieved via the \method{fetchone()}, \method{fetchmany()}, and
\method{fetchall()} methods.
\end{methoddesc}

\begin{methoddesc}[Cursor]{close}{}
Implements the DB-API 2.0 cursor \method{close()} method.

Cancels any pending results immediately.  Any operation on the cursor
after calling this method will raise an exception.

This method is called by the \method{__del__()} method.
\end{methoddesc}

\begin{methoddesc}[Cursor]{execute}{sql \optional{, params}}
Implements the DB-API 2.0 cursor \method{execute()} method.

Prepares a dynamic SQL command on the Sybase server and executes it.
The optional \var{params} argument is a sequence of parameters which
are bound as parameters to the dynamic SQL command.  Sybase uses
question mark (\code{?}) placeholders to specify which the parameters
will be used by the SQL command.

\begin{verbatim}
import Sybase
db = Sybase.connect('SYBASE', 'sa', '', 'pubs2')
c = db.cursor()
c.execute("select * from titles where price > ?", [15.00])
c.fetchall()
\end{verbatim}

The prepared dynamic SQL will be reused by the cursor if the same SQL
is passed in the \var{sql} argument.  This is most effective for
algorithms where the same operation is used, but different parameters
are bound to it (many times).

The method returns \code{None}.
\end{methoddesc}

\begin{methoddesc}[Cursor]{executemany}{sql, params_seq}
Implements the DB-API 2.0 cursor \method{executemany()} method.

Calls the \method{execute()} method for every parameter sequence in
the sequence passed as the \var{params_seq} argument.

The method returns \code{None}.
\end{methoddesc}

\begin{methoddesc}[Cursor]{fetchone}{}
Implements the DB-API 2.0 cursor \method{fetchone()} method.

Fetches the next row of a logical result and returns it as a tuple.
\code{None} is returned when no more rows are available in the logical
result.
\end{methoddesc}

\begin{methoddesc}[Cursor]{fetchmany}{\optional{size \code{= cursor.arraysize}}}
Implements the DB-API 2.0 cursor \method{fetchmany()} method.

Fetches the next set of rows of a logical result, returning a list of
tuples. An empty list is returned when no more rows are available.

The number of rows to fetch per call is specified by the optional
\var{size} argument.  If \var{size} is not supplied, the
\member{arraysize} member determines the number of rows to be
fetched.  The method will try to fetch the number of rows indicated by
the size parameter.  If this is not possible due to the specified
number of rows not being available, fewer rows will be returned.
\end{methoddesc}

\begin{methoddesc}[Cursor]{fetchall}{}
Implements the DB-API 2.0 cursor \method{fetchall()} method.

Fetches all remaining rows of a logical result returning them as a
list of tuples.
\end{methoddesc}

\begin{methoddesc}[Cursor]{nextset}{}
Implements the DB-API 2.0 cursor \method{nextset()} method.

Makes the cursor skip to the next logical result, discarding any
remaining rows from the current logical result.

If there are no more logical results, the method returns \code{None}.
Otherwise, it returns \code{1} and subsequent calls to the
\method{fetchone()}, \method{fetchmany()}, and \method{fetchall()}
methods will return rows from the next logical result.
\end{methoddesc}

\begin{memberdesc}[Cursor]{arraysize}
The DB-API 2.0 cursor \member{arraysize} member.

This read/write attribute specifies the number of rows to fetch at a
time with \method{fetchmany()}.  It defaults to \code{1} meaning to
fetch a single row at a time.
\end{memberdesc}

\begin{methoddesc}[Cursor]{setinputsizes}{size}
Implements the DB-API 2.0 cursor \method{setinputsizes()} method.

This method does nothing -- it is provided for DB-API compliance.
\end{methoddesc}

\begin{methoddesc}[Cursor]{setoutputsize}{size \optional{, column}}
Implements the DB-API 2.0 cursor \method{setoutputsize()} method.

This method does nothing -- it is provided for DB-API compliance.
\end{methoddesc}

\subsection{Bulkcopy Objects}

This is an extension of the DB-API 2.0 specification.

This object provides an interface to the Sybase bulkcopy
functionality.

\begin{methoddesc}[Bulkcopy]{rowxfer}{\optional{data}}
If the \class{Bulkcopy} object direction is \code{CS_BLK_IN} then the
sequence passed as the \var{data} argument is sent as one row to the
server.  If the direction is \code{CS_BLK_OUT} thenone row will be
returned from the server.
\end{methoddesc}

\begin{methoddesc}[Bulkcopy]{batch}{}
Marks a complete bulkcopy batch.  The number of rows transferred in
the batch is returned.
\end{methoddesc}

\begin{methoddesc}[Bulkcopy]{done}{}
Marks a complete bulkcopy operation.  The number of rows transferred
in the batch is returned.
\end{methoddesc}

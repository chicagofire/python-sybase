\subsection{CS_LOCALE Objects}

\class{CS_LOCALE} objects are a wrapper around the Sybase
\code{CS_LOCALE} structure.  The objects are created by calling the
\method{cs_loc_alloc()} method of a \class{CS_CONTEXT} object.

They have the following interface:

\begin{methoddesc}[CS_LOCALE]{cs_dt_info}{action, type \optional{, \ldots}}
Sets or retrieves datetime information of the locale object

When \var{action} is \code{CS_SET} a compatible \var{value} argument
must be supplied and the method returns the Sybase result code.  The
Sybase-CT \function{cs_dt_info()} function is called like this:

\begin{verbatim}
status = cs_dt_info(ctx, CS_SET, locale, type, CS_UNUSED,
                    &int_value, sizeof(int_value), &out_len);
\end{verbatim}

When \var{action} is \code{CS_GET} the method returns a tuple
containing the Sybase result code and a value.  When a string value is
requested an optional \var{item} argument can be passed which defaults
to \code{CS_UNUSED}.

The return result depends upon the value of the \var{type} argument.

\begin{longtable}{l|l|l}
\var{type} & need item? & return values \\
\hline
\code{CS_12HOUR}     & no  & \code{status, bool} \\
\code{CS_DT_CONVFMT} & no  & \code{status, int} \\
\code{CS_MONTH}      & yes & \code{status, string} \\
\code{CS_SHORTMONTH} & yes & \code{status, string} \\
\code{CS_DAYNAME}    & yes & \code{status, string} \\
\code{CS_DATEORDER}  & no  & \code{status, string} \\
\end{longtable}

The Sybase-CT \function{cs_dt_info()} function is called like this:

\begin{verbatim}
/* bool value */
status = cs_dt_info(ctx, CS_GET, locale, type, CS_UNUSED,
                    &bool_value, sizeof(bool_value), &out_len);

/* int value */
status = cs_dt_info(ctx, CS_GET, locale, type, CS_UNUSED,
                    &int_value, sizeof(int_value), &out_len);

/* string value */
status = cs_dt_info(ctx, CS_GET, locale, type, item,
                    str_buff, sizeof(str_buff), &buff_len);
\end{verbatim}
\end{methoddesc}

\begin{methoddesc}[CS_LOCALE]{cs_loc_drop}{}
Calls the Sybase \function{cs_loc_drop()} function like this:

\begin{verbatim}
status = cs_loc_drop(ctx, locale);
\end{verbatim}

Returns the Sybase result code.

This method will be automatically called when the \class{CS_LOCALE}
object is deleted.  Applications do not need to call the method.
\end{methoddesc}

\begin{methoddesc}[CS_LOCALE]{cs_locale}{action, type \optional{, value}}
Load the object with localisation values or retrieves the locale name
previously used to load the object.

When \var{action} is \code{CS_SET} a string \var{value} argument
must be supplied and the method returns the Sybase result code.  The
Sybase-CT \function{cs_locale()} function is called like this:

\begin{verbatim}
status = cs_locale(ctx, CS_SET, locale, type, value, CS_NULLTERM, NULL);
\end{verbatim}

The recognised values for \var{type} are:

\begin{longtable}{l}
\var{type} \\
\hline
\code{CS_LC_COLLATE}  \\
\code{CS_LC_CTYPE}    \\
\code{CS_LC_MESSAGE}  \\
\code{CS_LC_MONETARY} \\
\code{CS_LC_NUMERIC}  \\
\code{CS_LC_TIME}     \\
\code{CS_LC_ALL}      \\
\code{CS_SYB_LANG}    \\
\code{CS_SYB_CHARSET} \\
\code{CS_SYB_SORTORDER} \\
\code{CS_SYB_COLLATE}   \\
\code{CS_SYB_LANG_CHARSET} \\
\code{CS_SYB_TIME}     \\
\code{CS_SYB_MONETARY} \\
\code{CS_SYB_NUMERIC}  \\
\end{longtable}

When \var{action} is \code{CS_GET} the method returns a tuple
containing the Sybase result code and a locale name.  The
Sybase-CT \function{cs_locale()} function is called like this:

\begin{verbatim}
status = cs_locale(ctx, CS_GET, locale, type, str_buff, sizeof(str_buff), &str_len);
\end{verbatim}
\end{methoddesc}

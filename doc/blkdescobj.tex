\subsection{CS_BLKDESC Objects}

Calling the \method{blk_alloc()} method of a \class{CS_CONNECTION}
object will create a \class{CS_BLKDESC} object.  When the
\class{CS_BLKDESC} object is deallocated the Sybase
\function{blk_drop()} function will be called for the command.

\class{CS_BLKDESC} objects have the following interface:

\begin{methoddesc}[CS_BLKDESC]{blk_bind}{num, databuf}
Calls the Sybase \function{blk_bind()} function and returns the Sybase
result code.  The Sybase-CT \function{blk_bind()} function is called
like this:

\begin{verbatim}
status = blk_bind(blk, num, &datafmt, buffer->buff, buffer->copied, buffer->indicator);
\end{verbatim}
\end{methoddesc}

\begin{methoddesc}[CS_BLKDESC]{blk_describe}{num}
Calls the Sybase \function{blk_describe()} function and returns a
tuple containing the Sybase result code and a \class{CS_DATAFMT}
object which describes the column identified by \var{num}. \code{None}
is returned as the \class{CS_DATAFMT} object when the result code is
not \code{CS_SUCCEED}.

The Sybase \function{blk_describe()} function is called like this:

\begin{verbatim}
status = blk_describe(blk, num, &datafmt);
\end{verbatim}
\end{methoddesc}

\begin{methoddesc}[CS_BLKDESC]{blk_done}{type}
Calls the Sybase \function{blk_done()} function and returns a tuple
containing the Sybase result code and the number of rows copied in the
current batch.

The Sybase \function{blk_done()} function is called like this:

\begin{verbatim}
status = blk_done(blk, type, &num_rows);
\end{verbatim}
\end{methoddesc}

\begin{methoddesc}[CS_BLKDESC]{blk_drop}{}
Calls the Sybase \function{blk_drop()} function and returns the Sybase
result code.

The Sybase \function{blk_drop()} function is called like this:

\begin{verbatim}
status = blk_drop(blk);
\end{verbatim}

This method will be automatically called when the \class{CS_BLKDESC}
object is deleted.  Applications do not need to call the method.
\end{methoddesc}

\begin{methoddesc}[CS_BLKDESC]{blk_init}{direction, table}
Calls the Sybase \function{blk_init()} function and returns the Sybase
result code.

The Sybase \function{blk_init()} function is called like this:

\begin{verbatim}
status = blk_init(blk, direction, table, CS_NULLTERM);
\end{verbatim}
\end{methoddesc}

\begin{methoddesc}[CS_BLKDESC]{blk_props}{action, property \optional{, value}}
Sets, retrieves and clears properties of the bulk descriptor object.

When \var{action} is \code{CS_SET} a compatible \var{value} argument
must be supplied and the method returns the Sybase result code.  The
Sybase \function{blk_props()} function is called like this:

\begin{verbatim}
/* boolean property value */
status = blk_props(blk, CS_SET, property, &bool_value, CS_UNUSED, NULL);

/* int property value */
status = blk_props(blk, CS_SET, property, &int_value, CS_UNUSED, NULL);

/* numeric property value */
status = blk_props(blk, CS_SET, property, &numeric_value, CS_UNUSED, NULL);
\end{verbatim}

When \var{action} is \code{CS_GET} the method returns a tuple
containing the Sybase result code and the property value.  The Sybase
\function{blk_props()} function is called like this:

\begin{verbatim}
/* boolean property value */
status = blk_props(blk, CS_GET, property, &bool_value, CS_UNUSED, NULL);

/* int property value */
status = blk_props(blk, CS_GET, property, &int_value, CS_UNUSED, NULL);

/* numeric property value */
status = blk_props(blk, CS_GET, property, &numeric_value, CS_UNUSED, NULL);
\end{verbatim}

When \var{action} is \code{CS_CLEAR} the method returns the Sybase
result code.  The Sybase \function{blk_props()} function is called
like this:

\begin{verbatim}
status = blk_props(blk, CS_CLEAR, property, NULL, CS_UNUSED, NULL);
\end{verbatim}

The recognised properties are:

\begin{longtable}{l|l}
\var{property} & type \\
\hline
\code{BLK_IDENTITY}        & \code{bool} \\
\code{BLK_NOAPI_CHK}       & \code{bool} \\
\code{BLK_SENSITIVITY_LBL} & \code{bool} \\
\code{ARRAY_INSERT}        & \code{bool} \\
\code{BLK_SLICENUM}        & \code{int} \\
\code{BLK_IDSTARTNUM}      & \code{numeric} \\
\end{longtable}

For an explanation of the property values and get/set/clear semantics
please refer to the Sybase documentation.
\end{methoddesc}

\begin{methoddesc}[CS_BLKDESC]{blk_rowxfer}{}
Calls the Sybase \function{blk_rowxfer()} function and returns the
Sybase result code.

The Sybase \function{blk_rowxfer()} function is called like this:

\begin{verbatim}
status = blk_rowxfer(blk);
\end{verbatim}
\end{methoddesc}

\begin{methoddesc}[CS_BLKDESC]{blk_rowxfer_mult}{\optional{row_count}}
Calls the Sybase \function{blk_rowxfer_mult()} function and returns a
tuple containing the Sybase result code and the number of rows
transferred.

The Sybase \function{blk_rowxfer_mult()} function is called like this:

\begin{verbatim}
status = blk_rowxfer_mult(blk, &row_count);
\end{verbatim}
\end{methoddesc}

\begin{methoddesc}[CS_BLKDESC]{blk_textxfer}{\optional{str}}
Calls the Sybase \function{blk_textxfer()} function.  Depending on the
direction of the bulkcopy (established via the \method{blk_init()}
method) the method expects different arguments.

When \code{direction} \code{CS_BLK_IN} the \var{str} argument must be
supplied and method returns the Sybase result code.

The Sybase \function{blk_textxfer()} function is called like this:

\begin{verbatim}
status = blk_textxfer(blk, str, str_len, NULL);
\end{verbatim}

When \code{direction} \code{CS_BLK_OUT} the \var{str} argument must
not be present and method returns a tuple containing the Sybase result
code and a string.

The Sybase \function{blk_textxfer()} function is called like this:

\begin{verbatim}
status = blk_textxfer(blk, buff, sizeof(buff), &out_len);
\end{verbatim}
\end{methoddesc}

A simplistic program to bulkcopy a table from one server to another
server follows:

The first section contains the code to display client and server
messages in case of failure.

\begin{verbatim}
import sys
from sybasect import *

def print_msgs(conn, type):
    status, num_msgs = conn.ct_diag(CS_STATUS, type)
    if status != CS_SUCCEED:
        return
    for i in range(num_msgs):
        status, msg = conn.ct_diag(CS_GET, type, i + 1)
        if status != CS_SUCCEED:
            continue
        for attr in dir(msg):
            sys.stderr.write('%s: %s\n' % (attr, getattr(msg, attr)))

def die(conn, func):
    sys.stderr.write('%s failed!\n' % func)
    print_msgs(conn, CS_SERVERMSG_TYPE)
    print_msgs(conn, CS_CLIENTMSG_TYPE)
    sys.exit(1)
\end{verbatim}

The next section is fairly constant for all CT library programs.  A
library context is allocated and connections established.  The only
thing which is unique to bulk copy operations is setting the
\code{CS_BULK_LOGIN} option on the connection.

\begin{verbatim}
def init_db():
    status, ctx = cs_ctx_alloc()
    if status != CS_SUCCEED:
        raise 'cs_ctx_alloc'
    if ctx.ct_init(CS_VERSION_100) != CS_SUCCEED:
        raise 'ct_init'
    return ctx

def connect_db(ctx, server, user, passwd):
    status, conn = ctx.ct_con_alloc()
    if status != CS_SUCCEED:
        raise 'ct_con_alloc'
    if conn.ct_diag(CS_INIT) != CS_SUCCEED:
        die(conn, 'ct_diag')
    if conn.ct_con_props(CS_SET, CS_USERNAME, user) != CS_SUCCEED:
        die(conn, 'ct_con_props CS_USERNAME')
    if conn.ct_con_props(CS_SET, CS_PASSWORD, passwd) != CS_SUCCEED:
        die(conn, 'ct_con_props CS_PASSWORD')
    if conn.ct_con_props(CS_SET, CS_BULK_LOGIN, 1) != CS_SUCCEED:
        die(conn, 'ct_con_props CS_BULK_LOGIN')
    if conn.ct_connect(server) != CS_SUCCEED:
        die(conn, 'ct_connect')
    return conn
\end{verbatim}

The next segment allocates bulkcopy descriptors, data buffers, and
binds the data buffers to the bulk copy descriptors.  The same buffers
are used for copying out and copying in - not bad.  Note that for array
binding we need to use loose packing for copy in; hence the line
setting the \code{format} member of \code{Databuf} \code{CS_DATAFMT} to
\code{CS_BLK_ARRAY_MAXLEN}.  Without this the bulkcopy operation
assumes tight packing and the data is corrupted on input.

\begin{verbatim}
def alloc_bcp(conn, dirn, table):
    status, blk = conn.blk_alloc()
    if status != CS_SUCCEED:
        die(conn, 'blk_alloc')
    if blk.blk_init(dirn, table) != CS_SUCCEED:
        die(conn, 'blk_init')
    return blk

def alloc_bufs(bcp, num):
    bufs = []
    while 1:
        status, fmt = bcp.blk_describe(len(bufs) + 1)
        if status != CS_SUCCEED:
            break
        fmt.count = num
        bufs.append(DataBuf(fmt))
    return bufs

def bcp_bind(bcp, bufs):
    for i in range(len(bufs)):
        buf = bufs[i]
        if bcp.direction == CS_BLK_OUT:
            buf.format = 0
        else:
            buf.format = CS_BLK_ARRAY_MAXLEN
        if bcp.blk_bind(i + 1, buf) != CS_SUCCEED:
            die(bcp.conn, 'blk_bind')
\end{verbatim}

This next section actually performs the bulkcopy.  Note that there is
no attempt to deal with BLOB columns.

\begin{verbatim}
def bcp_copy(src, dst, batch_size):
    total = batch = 0
    while 1:
        status, num_rows = src.blk_rowxfer_mult()
        if status == CS_END_DATA:
            break
        if status != CS_SUCCEED:
            die(src, 'blk_rowxfer_mult out')
        status, dummy = dst.blk_rowxfer_mult(num_rows)
        if status != CS_SUCCEED:
            die(src, 'blk_rowxfer_mult in')
        batch = batch + num_rows
        if batch >= batch_size:
            total = total + batch
            batch = 0
            src.blk_done(CS_BLK_BATCH)
            dst.blk_done(CS_BLK_BATCH)
            print 'batch - %d rows transferred' % total

    status, num_rows = src.blk_done(CS_BLK_ALL)
    status, num_rows = dst.blk_done(CS_BLK_ALL)
    return total + batch
\end{verbatim}

Finally the code which drives the whole process.

\begin{verbatim}
ctx = init_db()
src_conn = connect_db(ctx, 'drama', 'sa', '')
dst_conn = connect_db(ctx, 'SYBASE', 'sa', '')
src = alloc_bcp(src_conn, CS_BLK_OUT, 'pubs2.dbo.authors')
dst = alloc_bcp(dst_conn, CS_BLK_IN,  'test.dbo.authors')

bufs = alloc_bufs(src, 5)
bcp_bind(src, bufs)
bcp_bind(dst, bufs)

total = bcp_copy(src, dst, 10)
print 'all done - %d rows transferred' % total
\end{verbatim}

\subsection{CS_COMMAND Objects}

Calling the \method{ct_cmd_alloc()} method of a \class{CS_CONNECTION}
object will create a \class{CS_COMMAND} object.  When the
\class{CS_COMMAND} object is deallocated the Sybase
\function{ct_cmd_drop()} function will be called for the command.

\class{CS_COMMAND} objects have the following interface:

\begin{memberdesc}[CS_COMMAND]{is_eed}
A read only integer which indicates when the \class{CS_COMMAND} object
is an extended error data command structure.
\end{memberdesc}

\begin{memberdesc}[CS_COMMAND]{conn}
This is a read only reference to the parent \class{CS_CONNECTION}
object.  This prevents the connection from being dropped while the
command still exists.
\end{memberdesc}

\begin{memberdesc}[CS_COMMAND]{strip}
An integer which controls right whitespace stripping of \code{char}
columns.  The default value is inherited from the parent connection
when the command is created.
\end{memberdesc}

\begin{memberdesc}[CS_COMMAND]{debug}
An integer which controls printing of debug messages to \code{stderr}.
The default value is inherited from the parent connection when the
command is created.
\end{memberdesc}

\begin{methoddesc}[CS_COMMAND]{ct_bind}{num, datafmt}
Calls the Sybase-CT \function{ct_bind()} function and returns a tuple
containing the Sybase result code and a \class{DataBuf} object which
is used to retrieve data from the column identified by \var{num}.
\code{None} is returned as the \class{DataBuf} object when the result
code is not \code{CS_SUCCEED}.  The Sybase-CT \function{ct_bind()}
function is called like this:

\begin{verbatim}
status = ct_bind(cmd, num, &datafmt, databuf->buff, databuf->copied, databuf->indicator);
\end{verbatim}

See the description of the \method{ct_describe()} method for an
example of how to use this method in Python.
\end{methoddesc}

\begin{methoddesc}[CS_COMMAND]{ct_cancel}{type}
Calls the Sybase \function{ct_cancel()} function like this:

\begin{verbatim}
status = ct_cancel(NULL, cmd, type);
\end{verbatim}

Returns the Sybase result code.
\end{methoddesc}

\begin{methoddesc}[CS_COMMAND]{ct_cmd_drop}{}
Calls the Sybase-CT \function{ct_cmd_drop()} function like this:

\begin{verbatim}
status = ct_cmd_drop(cmd);
\end{verbatim}

Returns the Sybase result code.

This method will be automatically called when the \class{CS_COMMAND}
object is deleted.  Applications do not need to call the method.
\end{methoddesc}

\begin{methoddesc}[CS_COMMAND]{ct_command}{type \optional{, \ldots}}
Calls the Sybase-CT \function{ct_command()} function and returns the
result code.  The \var{type} argument determines the type and number
of additional arguments.

When \var{type} is \code{CS_LANG_CMD} the method must be called like
this:

\code{ct_command(CS_LANG_CMD, \var{sql_text} \optional{, \var{option \code{= CS_UNUSED}}})}

Then the Sybase-CT \function{ct_command()} function is called like
this:

\begin{verbatim}
status = ct_command(cmd, CS_LANG_CMD, sql_text, CS_NULLTERM, option);
\end{verbatim}

When \var{type} is \code{CS_RPC_CMD} the method must be called like
this:

\code{ct_command(CS_RPC_CMD, \var{proc_name} \optional{, \var{option \code{= CS_UNUSED}}})}

Then the Sybase-CT \function{ct_command()} function is called like
this:

\begin{verbatim}
status = ct_command(cmd, CS_RPC_CMD, proc_name, CS_NULLTERM, option);
\end{verbatim}

When \var{type} is \code{CS_MSG_CMD} the method must be called like
this:

\code{ct_command(CS_MSG_CMD, \var{msg_num})}

Then the Sybase-CT \function{ct_command()} function is called like
this:

\begin{verbatim}
status = ct_command(cmd, CS_MSG_CMD, &msg_num, CS_UNUSED, CS_UNUSED);
\end{verbatim}

When \var{type} is \code{CS_PACKAGE_CMD} the method must be called like
this:

\code{ct_command(CS_PACKAGE_CMD, \var{pkg_name})}

Then the Sybase-CT \function{ct_command()} function is called like
this:

\begin{verbatim}
status = ct_command(cmd, CS_PACKAGE_CMD, pkg_name, CS_NULLTERM, CS_UNUSED);
\end{verbatim}

When \var{type} is \code{CS_SEND_DATA_CMD} the method must be called like
this:

\code{ct_command(CS_SEND_DATA_CMD)}

Then the Sybase-CT \function{ct_command()} function is called like
this:

\begin{verbatim}
status = ct_command(cmd, CS_SEND_DATA_CMD, NULL, CS_UNUSED, CS_COLUMN_DATA);
\end{verbatim}

For an explanation of the argument semantics please refer to the
Sybase documentation.
\end{methoddesc}

\begin{methoddesc}[CS_COMMAND]{ct_cursor}{type \optional{, \ldots}}
Calls the Sybase \function{ct_cursor()} function and returns the
result code.  The \var{type} argument determines the type and number
of additional arguments.

When \var{type} is \code{CS_CURSOR_DECLARE} the method must be called like
this:

\code{ct_cursor(CS_CURSOR_DECLARE, \var{cursor_id}, \var{sql_text} \optional{, \var{option \code{= CS_UNUSED}}})}

Then the Sybase-CT \function{ct_cursor()} function is called like
this:

\begin{verbatim}
status = ct_cursor(cmd, CS_CURSOR_DECLARE, cursor_id, CS_NULLTERM, sql_text, CS_NULLTERM, option);
\end{verbatim}

When \var{type} is \code{CS_CURSOR_UPDATE} the method must be called like
this:

\code{ct_cursor(CS_CURSOR_UPDATE, \var{table_name}, \var{sql_text} \optional{, \var{option \code{= CS_UNUSED}}})}

Then the Sybase-CT \function{ct_cursor()} function is called like
this:

\begin{verbatim}
status = ct_cursor(cmd, CS_CURSOR_UPDATE, table_name, CS_NULLTERM, sql_text, CS_NULLTERM, option);
\end{verbatim}

When \var{type} is \code{CS_CURSOR_OPTION} the method must be called like
this:

\code{ct_cursor(CS_CURSOR_OPTION \optional{, \var{option \code{= CS_UNUSED}}})}

Then the Sybase-CT \function{ct_cursor()} function is called like
this:

\begin{verbatim}
status = ct_cursor(cmd, CS_CURSOR_OPTION, NULL, CS_UNUSED, NULL, CS_UNUSED, option);
\end{verbatim}

When \var{type} is \code{CS_CURSOR_OPEN} the method must be called like
this:

\code{ct_cursor(CS_CURSOR_OPEN \optional{, \var{option \code{= CS_UNUSED}}})}

Then the Sybase-CT \function{ct_cursor()} function is called like
this:

\begin{verbatim}
status = ct_cursor(cmd, CS_CURSOR_OPEN, NULL, CS_UNUSED, NULL, CS_UNUSED, option);
\end{verbatim}

When \var{type} is \code{CS_CURSOR_CLOSE} the method must be called like
this:

\code{ct_cursor(CS_CURSOR_CLOSE \optional{, \var{option \code{= CS_UNUSED}}})}

Then the Sybase-CT \function{ct_cursor()} function is called like
this:

\begin{verbatim}
status = ct_cursor(cmd, CS_CURSOR_CLOSE, NULL, CS_UNUSED, NULL, CS_UNUSED, option);
\end{verbatim}

When \var{type} is \code{CS_CURSOR_ROWS} the method must be called like
this:

\code{ct_cursor(CS_CURSOR_ROWS, \var{num_rows})}

Then the Sybase-CT \function{ct_cursor()} function is called like
this:

\begin{verbatim}
status = ct_cursor(cmd, CS_CURSOR_ROWS, NULL, CS_UNUSED, NULL, CS_UNUSED, num_rows);
\end{verbatim}

When \var{type} is \code{CS_CURSOR_DELETE} the method must be called like
this:

\code{ct_cursor(CS_CURSOR_DELETE, \var{table_name})}

Then the Sybase-CT \function{ct_cursor()} function is called like
this:

\begin{verbatim}
status = ct_cursor(cmd, CS_CURSOR_DELETE, table_name, CS_NULLTERM, NULL, CS_UNUSED, CS_UNUSED);
\end{verbatim}

When \var{type} is \code{CS_CURSOR_DEALLOC} the method must be called like
this:

\code{ct_cursor(CS_CURSOR_DEALLOC)}

Then the Sybase-CT \function{ct_cursor()} function is called like
this:

\begin{verbatim}
status = ct_cursor(cmd, CS_CURSOR_DEALLOC, NULL, CS_UNUSED, NULL, CS_UNUSED, CS_UNUSED);
\end{verbatim}

For an explanation of the argument semantics please refer to the
Sybase documentation.

The \texttt{cursor_sel.py}, \texttt{cursor_upd.py}, and
\texttt{dynamic_cur.py} example programs contain examples of this
function.
\end{methoddesc}

\begin{methoddesc}[CS_COMMAND]{ct_data_info}{action, \ldots}
Sets and retrieves column IO descriptors.

When \var{action} is \code{CS_SET} the method must be called like
this:

\code{ct_data_info(CS_SET, \var{iodesc})}

Returns the Sybase result code.  The Sybase-CT
\function{ct_data_info()} function is called like this:

\begin{verbatim}
status = ct_data_info(cmd, CS_SET, CS_UNUSED, &iodesc);
\end{verbatim}

When \var{action} is \code{CS_GET} the method must be called like
this:

\code{ct_data_info(CS_SET, \var{num})}

Returns a tuple containing the Sybase result code and an
\code{CS_IODESC} object.  If the result code is not \code{CS_SUCCEED}
then \code{None} is returned as the \code{CS_IODESC} object.  The
Sybase-CT \function{ct_data_info()} function is called like this:

\begin{verbatim}
status = ct_data_info(cmd, CS_GET, num, &iodesc);
\end{verbatim}

For an explanation of the argument semantics please refer to the
Sybase documentation.

The \texttt{mult_text.py} example program contains examples of this
function.
\end{methoddesc}

\begin{methoddesc}[CS_COMMAND]{ct_describe}{num}
Calls the Sybase \function{ct_describe()} function and returns a tuple
containing the Sybase result code and a \class{CS_DATAFMT} object
which describes the column identified by \var{num}. \code{None} is
returned as the \class{CS_DATAFMT} object when the result code is not
\code{CS_SUCCEED}.

The Sybase-CT \function{ct_describe()} function is called like this:

\begin{verbatim}
status = ct_describe(cmd, num, &datafmt);
\end{verbatim}

The following constructs a list of buffers for retrieving a number of
rows from a command object.

\begin{verbatim}
def row_bind(cmd, count = 1):
    status, num_cols = cmd.ct_res_info(CS_NUMDATA)
    if status != CS_SUCCEED:
        raise 'ct_res_info'
    bufs = []
    for i in range(num_cols):
        status, fmt = cmd.ct_describe(i + 1)
        if status != CS_SUCCEED:
            raise 'ct_describe'
        fmt.count = count
        status, buf = cmd.ct_bind(i + 1, fmt)
        if status != CS_SUCCEED:
            raise 'ct_bind'
        bufs.append(buf)
    return bufs
\end{verbatim}
\end{methoddesc}

\begin{methoddesc}[CS_COMMAND]{ct_dynamic}{type, \ldots}
Calls the Sybase \function{ct_dynamic()} function and returns the
result code.  The \var{type} argument determines the type and number
of additional arguments.

When \var{type} is \code{CS_CURSOR_DECLARE} the method must be called
like this:

\code{ct_dynamic(CS_CURSOR_DECLARE, \var{dyn_id}, \var{cursor_id})}

Then the Sybase-CT \function{ct_dynamic()} function is called like
this:

\begin{verbatim}
status = ct_dynamic(cmd, CS_CURSOR_DECLARE, dyn_id, CS_NULLTERM, cursor_id, CS_NULLTERM);
\end{verbatim}

When \var{type} is \code{CS_DEALLOC} the method must be called like
this:

\code{ct_dynamic(CS_DEALLOC, \var{dyn_id})}

Then the Sybase-CT \function{ct_dynamic()} function is called like
this:

\begin{verbatim}
status = ct_dynamic(cmd, CS_DEALLOC, dyn_id, CS_NULLTERM, NULL, CS_UNUSED);
\end{verbatim}

When \var{type} is \code{CS_DESCRIBE_INPUT} the method must be called
like this:

\code{ct_dynamic(CS_DESCRIBE_INPUT, \var{dyn_id})}

Then the Sybase-CT \function{ct_dynamic()} function is called like
this:

\begin{verbatim}
status = ct_dynamic(cmd, CS_DESCRIBE_INPUT, dyn_id, CS_NULLTERM, NULL, CS_UNUSED);
\end{verbatim}

When \var{type} is \code{CS_DESCRIBE_OUTPUT} the method must be called
like this:

\code{ct_dynamic(CS_DESCRIBE_OUTPUT, \var{dyn_id})}

Then the Sybase-CT \function{ct_dynamic()} function is called like
this:

\begin{verbatim}
status = ct_dynamic(cmd, CS_DESCRIBE_OUTPUT, dyn_id, CS_NULLTERM, NULL, CS_UNUSED);
\end{verbatim}

When \var{type} is \code{CS_EXECUTE} the method must be called like
this:

\code{ct_dynamic(CS_EXECUTE, \var{dyn_id})}

Then the Sybase-CT \function{ct_dynamic()} function is called like
this:

\begin{verbatim}
status = ct_dynamic(cmd, CS_EXECUTE, dyn_id, CS_NULLTERM, NULL, CS_UNUSED);
\end{verbatim}

When \var{type} is \code{CS_EXEC_IMMEDIATE} the method must be called
like this:

\code{ct_dynamic(CS_EXEC_IMMEDIATE, \var{sql_text})}

Then the Sybase-CT \function{ct_dynamic()} function is called like
this:

\begin{verbatim}
status = ct_dynamic(cmd, CS_EXEC_IMMEDIATE, NULL, CS_UNUSED, sql_text, CS_NULLTERM);
\end{verbatim}

When \var{type} is \code{CS_EXECUTE} the method must be called like
this:

\code{ct_dynamic(CS_PREPARE, \var{dyn_id}, \var{sql_text})}

Then the Sybase-CT \function{ct_dynamic()} function is called like
this:

\begin{verbatim}
status = ct_dynamic(cmd, CS_PREPARE, dyn_id, CS_NULLTERM, sql_text, CS_NULLTERM);
\end{verbatim}

For an explanation of the argument semantics please refer to the
Sybase documentation.

The \texttt{dynamic_cur.py}, and \texttt{dynamic_ins.py} example
programs contain examples of this function.
\end{methoddesc}

\begin{methoddesc}[CS_COMMAND]{ct_fetch}{}
Calls the Sybase \function{ct_fetch()} function and returns a tuple
containing the Sybase result code and the number of rows read (for
array binding).

The Sybase-CT \function{ct_fetch()} function is called like this:

\begin{verbatim}
status = ct_fetch(cmd, CS_UNUSED, CS_UNUSED, CS_UNUSED, &rows_read);
\end{verbatim}
\end{methoddesc}

\begin{methoddesc}[CS_COMMAND]{ct_get_data}{num, databuf}
Calls the Sybase \function{ct_get_data()} function and returns a tuple
containing the Sybase result code and the length of the data for item
number \var{num} which was read into the \class{DataBuf} object in the
\var{databuf} argument.

The Sybase-CT \function{ct_get_data()} function is called like this:

\begin{verbatim}
status = ct_get_data(cmd, num, databuf->buff, databuf->fmt.maxlength, databuf->copied);
\end{verbatim}

The following will retrieve the contents of a BLOB column:

\begin{verbatim}
def get_blob_column(cmd, col):
    fmt = CS_DATAFMT()
    fmt.maxlength = 32768
    buf = DataBuf(fmt)
    parts = []
    while 1:
        status, count = cmd.ct_get_data(col, buf)
        if count:
            parts.append(buf[0])
        if status != CS_SUCCEED:
            break
    return string.join(parts, '')
\end{verbatim}
\end{methoddesc}

\begin{methoddesc}[CS_COMMAND]{ct_param}{param}
Calls the Sybase \function{ct_param()} function and returns the Sybase
result code.

The \var{param} argument is usually an instance of the \class{DataBuf}
class.  A \class{CS_DATAFMT} object can be used in a cursor declare
context to define the format of the host variable.

When \var{param} is a \class{DataBuf} the Sybase-CT
\function{ct_param()} function is called like this:

\begin{verbatim}
status = ct_param(cmd, &databuf->fmt, databuf->buff, databuf->copied[0], databuf->indicator[0]);
\end{verbatim}

When \var{param} is a \class{CS_DATAFMT} the Sybase-CT
\function{ct_param()} function is called like this:

\begin{verbatim}
status = ct_param(cmd, &datafmt, NULL, CS_UNUSED, CS_UNUSED);
\end{verbatim}

The semantics of the \class{CS_DATAFMT} attributes are quite complex.
Please refer to the Sybase documentation.
\end{methoddesc}

\begin{methoddesc}[CS_COMMAND]{ct_res_info}{type}
Calls the Sybase \function{ct_res_info()} function.  The return result
depends upon the value of the \var{type} argument.

\begin{longtable}{l|l}
\var{type} & return values \\
\hline
\code{CS_BROWSE_INFO}   & \code{status, bool} \\
\code{CS_CMD_NUMBER}    & \code{status, int} \\
\code{CS_MSGTYPE}       & \code{status, int} \\
\code{CS_NUM_COMPUTES}  & \code{status, int} \\
\code{CS_NUMDATA}       & \code{status, int} \\
\code{CS_NUMORDER_COLS} & \code{status, int} \\
\code{CS_ORDERBY_COLS}  & \code{status, list of int} \\
\code{CS_ROW_COUNT}     & \code{status, int} \\
\code{CS_TRANS_STATE}   & \code{status, int} \\
\end{longtable}

Depending on type the Sybase-CT \function{ct_res_info()} function is
called like this:

\begin{verbatim}
status = ct_res_info(cmd, CS_BROWSE_INFO, &bool_val, CS_UNUSED, NULL);

status = ct_res_info(cmd, CS_MSGTYPE, &ushort_val, CS_UNUSED, NULL);

status = ct_res_info(cmd, CS_CMD_NUMBER, &int_val, CS_UNUSED, NULL);

status = ct_res_info(cmd, CS_NUM_COMPUTES, &int_val, CS_UNUSED, NULL);

status = ct_res_info(cmd, CS_NUMDATA, &int_val, CS_UNUSED, NULL);

status = ct_res_info(cmd, CS_NUMORDERCOLS, &int_val, CS_UNUSED, NULL);

status = ct_res_info(cmd, CS_ROW_COUNT, &int_val, CS_UNUSED, NULL);

status = ct_res_info(cmd, CS_TRANS_STATE, &int_val, CS_UNUSED, NULL);

status = ct_res_info(cmd, CS_NUMORDERCOLS, &int_val, CS_UNUSED, NULL);
status = ct_res_info(cmd, CS_ORDERBY_COLS, col_nums, sizeof(*col_nums) * int_val, NULL);
\end{verbatim}
\end{methoddesc}

\begin{methoddesc}[CS_COMMAND]{ct_results}{}
Calls the Sybase \function{ct_results()} function and returns a tuple
containing the Sybase result code and the result type returned by the
Sybase function.

The Sybase-CT \function{ct_results()} function is called like this:

\begin{verbatim}
status = ct_results(cmd, &result);
\end{verbatim}
\end{methoddesc}

\begin{methoddesc}[CS_COMMAND]{ct_send}{}
Calls the Sybase \function{ct_send()} function and returns the Sybase
result code.

The Sybase-CT \function{ct_send()} function is called like this:

\begin{verbatim}
status = ct_send(cmd);
\end{verbatim}
\end{methoddesc}

\begin{methoddesc}[CS_COMMAND]{ct_send_data}{databuf}
Calls the Sybase \function{ct_send_data()} function and returns the
Sybase result code.  The \var{databuf} argument must be a
\class{DataBuf} object.

The Sybase-CT \function{ct_send_data()} function is called like this:

\begin{verbatim}
status = ct_send_data(cmd, databuf->buff, databuf->copied[0]);
\end{verbatim}
\end{methoddesc}

\begin{methoddesc}[CS_COMMAND]{ct_setparam}{databuf}
Calls the Sybase \function{ct_setparam()} function and returns the
Sybase result code.  The \var{databuf} argument must be a
\class{DataBuf} object.

The Sybase-CT \function{ct_setparam()} function is called like this:

\begin{verbatim}
status = ct_setparam(cmd, &databuf->fmt, databuf->buff, databuf->copied, databuf->indicator);
\end{verbatim}
\end{methoddesc}

\subsection{CS_CONNECTION Objects}

Calling the \method{ct_con_alloc()} method of a \class{CS_CONTEXT}
object will create a \class{CS_CONNECTION} object.  When the
\class{CS_CONNECTION} object is deallocated the Sybase
\function{ct_con_drop()} function will be called for the connection.

\class{CS_CONNECTION} objects have the following interface:

\begin{memberdesc}[CS_CONNECTION]{ctx}
This is a read only reference to the parent \class{CS_CONTEXT} object.
This prevents the context from being dropped while the connection
still exists.
\end{memberdesc}

\begin{memberdesc}[CS_CONNECTION]{strip}
An integer which controls right whitespace stripping of \code{char}
columns.  The default value is zero.
\end{memberdesc}

\begin{memberdesc}[CS_CONNECTION]{debug}
An integer which controls printing of debug messages to the debug file
established by the \function{set_debug()} function.  The default value
is inherited from the \code{CS_CONTEXT} object.
\end{memberdesc}

\begin{methoddesc}[CS_CONNECTION]{ct_diag}{operation \optional{, \ldots}}
Manage Sybase error messages for the connection.

When \var{operation} is \code{CS_INIT} a single argument is accepted
and the Sybase result code is returned.  The Sybase
\function{ct_diag()} function is called like this:

\begin{verbatim}
status = ct_diag(conn, CS_INIT, CS_UNUSED, CS_UNUSED, NULL);
\end{verbatim}

When \var{operation} is \code{CS_MSGLIMIT} two additional arguments
are expected; \var{type} and \var{num}.  The Sybase result code is
returned.  The Sybase \function{ct_diag()} function is called like this:

\begin{verbatim}
status = ct_diag(conn, CS_MSGLIMIT, type, CS_UNUSED, &num);
\end{verbatim}

When \var{operation} is \code{CS_CLEAR} an additional \var{type}
argument is accepted and the Sybase result code is returned.  The
Sybase \function{ct_diag()} function is called like this:

\begin{verbatim}
status = ct_diag(conn, CS_CLEAR, type, CS_UNUSED, NULL);
\end{verbatim}

When \var{operation} is \code{CS_GET} two additional arguments are
expected; \var{type} and \var{index}.  A tuple is returned which
contains the Sybase result code and the requested \class{CS_SERVERMSG}
or \class{CS_CLIENTMSG} message.  \code{None} is returned as the
message object when the result code is not \code{CS_SUCCEED}.  The
Sybase \function{ct_diag()} function is called like this:

\begin{verbatim}
status = ct_diag(conn, CS_GET, type, index, &msg);
\end{verbatim}

When \var{operation} is \code{CS_STATUS} an additional \var{type}
argument is accepted.  A tuple is returned which contains the Sybase
result code and the number of messages available for retrieval.  The
Sybase \function{ct_diag()} function is called like this:

\begin{verbatim}
status = ct_diag(conn, CS_STATUS, type, CS_UNUSED, &num);
\end{verbatim}

When \var{operation} is \code{CS_EED_CMD} two additional arguments are
expected; \var{type} and \var{index}.  A tuple is returned which
contains the Sybase result code and a \class{CS_COMMAND} object which
is used to retrieve extended error data.  The Sybase
\function{ct_diag()} function is called like this:

\begin{verbatim}
status = ct_diag(conn, CS_EED_CMD, type, index, &eed);
\end{verbatim}

The following will retrieve and print all messages from a connection.

\begin{verbatim}
def print_msgs(conn, type):
    status, num_msgs = conn.ct_diag(CS_STATUS, type)
    if status != CS_SUCCEED:
        return
    for i in range(num_msgs):
        status, msg = conn.ct_diag(CS_GET, type, i + 1)
        if status != CS_SUCCEED:
            continue
        for attr in dir(msg):
            print '%s: %s' % (attr, getattr(msg, attr))

def print_all_msgs(conn):
    print_msgs(conn, CS_SERVERMSG_TYPE)
    print_msgs(conn, CS_CLIENTMSG_TYPE)
    conn.ct_diag(CS_CLEAR, CS_ALLMSG_TYPE)
\end{verbatim}
\end{methoddesc}

\begin{methoddesc}[CS_CONNECTION]{ct_cancel}{type}
Calls the Sybase \function{ct_cancel()} function and returns the
Sybase result code.  The Sybase \function{ct_cancel()} function is
called like this:

\begin{verbatim}
status = ct_cancel(conn, NULL, type);
\end{verbatim}
\end{methoddesc}

\begin{methoddesc}[CS_CONNECTION]{ct_connect}{\optional{server \code{= None}}}
Calls the Sybase \function{ct_connect()} function and returns the
Sybase result code.  The Sybase \function{ct_connect()} function is
called like this:

\begin{verbatim}
status = ct_connect(conn, server, CS_NULLTERM);
\end{verbatim}

When no \var{server} argument is supplied the Sybase
\function{ct_connect()} function is called like this:

\begin{verbatim}
status = ct_connect(conn, NULL, 0);
\end{verbatim}
\end{methoddesc}

\begin{methoddesc}[CS_CONNECTION]{ct_cmd_alloc}{}
Allocates and returns a new \class{CS_COMMAND} object which is used to
send commands over the connection.  Calls the Sybase-CT
\function{ct_callback()} function like this:

\begin{verbatim}
status = ct_cmd_alloc(conn, &cmd);
\end{verbatim}

The result is a tuple containing the Sybase result code and a new
instance of the \class{CS_COMMAND} class. \code{None} is returned as
the \class{CS_COMMAND} object when the result code is not
\code{CS_SUCCEED}.
\end{methoddesc}

\begin{methoddesc}[CS_CONNECTION]{blk_alloc}{\optional{version \code{= BLK_VERSION_100}}}
Allocates and returns a new \class{CS_BLKDESC} object which is used to
perform bulkcopy over the connection.  Calls the Sybase
\function{blk_alloc()} function like this:

\begin{verbatim}
status = blk_alloc(conn, version, &blk);
\end{verbatim}

The result is a tuple containing the Sybase result code and a new
instance of the \class{CS_BLKDESC} class. \code{None} is returned as
the \class{CS_BLKDESC} object when the result code is not
\code{CS_SUCCEED}.
\end{methoddesc}

\begin{methoddesc}[CS_CONNECTION]{ct_close}{\optional{option \code{ = CS_UNUSED}}}
Calls the Sybase \function{ct_close()} function like this:

\begin{verbatim}
status = ct_close(conn, option);
\end{verbatim}

Returns the Sybase result code.
\end{methoddesc}

\begin{methoddesc}[CS_CONNECTION]{ct_con_drop}{}
Calls the Sybase \function{ct_con_drop()} function like this:

\begin{verbatim}
status = ct_con_drop(conn);
\end{verbatim}

Returns the Sybase result code.

This method will be automatically called when the \class{CS_CONNECTION}
object is deleted.  Applications do not need to call the method.
\end{methoddesc}

\begin{methoddesc}[CS_CONNECTION]{ct_con_props}{action, property \optional{, value}}
Sets, retrieves and clears properties of the connection object.

When \var{action} is \code{CS_SET} a compatible \var{value} argument
must be supplied and the method returns the Sybase result code.  The
Sybase-CT \function{ct_con_props()} function is called like this:

\begin{verbatim}
/* boolean property value */
status = ct_con_props(conn, CS_SET, property, &bool_value, CS_UNUSED, NULL);

/* int property value */
status = ct_con_props(conn, CS_SET, property, &int_value, CS_UNUSED, NULL);

/* string property value */
status = ct_con_props(conn, CS_SET, property, str_value, CS_NULLTERM, NULL);
\end{verbatim}

When \var{action} is \code{CS_GET} the method returns a tuple
containing the Sybase result code and the property value.  The
Sybase-CT \function{ct_con_props()} function is called like this:

\begin{verbatim}
/* boolean property value */
status = ct_con_props(conn, CS_GET, property, &bool_value, CS_UNUSED, NULL);

/* int property value */
status = ct_con_props(conn, CS_GET, property, &int_value, CS_UNUSED, NULL);

/* string property value */
status = ct_con_props(conn, CS_GET, property, str_buff, sizeof(str_buff), &buff_len);
\end{verbatim}

When \var{action} is \code{CS_CLEAR} the method returns the Sybase
result code.  The Sybase-CT \function{ct_con_props()} function is
called like this:

\begin{verbatim}
status = ct_con_props(conn, CS_CLEAR, property, NULL, CS_UNUSED, NULL);
\end{verbatim}

The recognised properties are:

\begin{longtable}{l|l}
\var{property} & type \\
\hline
\code{CS_ANSI_BINDS}           & \code{bool} \\
\code{CS_ASYNC_NOTIFS}         & \code{bool} \\
\code{CS_BULK_LOGIN}           & \code{bool} \\
\code{CS_CHARSETCNV}           & \code{bool} \\
\code{CS_CONFIG_BY_SERVERNAME} & \code{bool} \\
\code{CS_DIAG_TIMEOUT}         & \code{bool} \\
\code{CS_DISABLE_POLL}         & \code{bool} \\
\code{CS_DS_COPY}              & \code{bool} \\
\code{CS_DS_EXPANDALIAS}       & \code{bool} \\
\code{CS_DS_FAILOVER}          & \code{bool} \\
\code{CS_EXPOSE_FMTS}          & \code{bool} \\
\code{CS_EXTERNAL_CONFIG}      & \code{bool} \\
\code{CS_EXTRA_INF}            & \code{bool} \\
\code{CS_HIDDEN_KEYS}          & \code{bool} \\
\code{CS_LOGIN_STATUS}         & \code{bool} \\
\code{CS_NOCHARSETCNV_REQD}    & \code{bool} \\
\code{CS_SEC_APPDEFINED}       & \code{bool} \\
\code{CS_SEC_CHALLENGE}        & \code{bool} \\
\code{CS_SEC_CHANBIND}         & \code{bool} \\
\code{CS_SEC_CONFIDENTIALITY}  & \code{bool} \\
\code{CS_SEC_DATAORIGIN}       & \code{bool} \\
\code{CS_SEC_DELEGATION}       & \code{bool} \\
\code{CS_SEC_DETECTREPLAY}     & \code{bool} \\
\code{CS_SEC_DETECTSEQ}        & \code{bool} \\
\code{CS_SEC_ENCRYPTION}       & \code{bool} \\
\code{CS_SEC_INTEGRITY}        & \code{bool} \\
\code{CS_SEC_MUTUALAUTH}       & \code{bool} \\
\code{CS_SEC_NEGOTIATE}        & \code{bool} \\
\code{CS_SEC_NETWORKAUTH}      & \code{bool} \\

\code{CS_CON_STATUS}           & \code{int} \\
\code{CS_LOOP_DELAY}           & \code{int} \\
\code{CS_RETRY_COUNT}          & \code{int} \\
\code{CS_NETIO}                & \code{int} \\
\code{CS_TEXTLIMIT}            & \code{int} \\
\code{CS_DS_SEARCH}            & \code{int} \\
\code{CS_DS_SIZELIMIT}         & \code{int} \\
\code{CS_DS_TIMELIMIT}         & \code{int} \\
\code{CS_ENDPOINT}             & \code{int} \\
\code{CS_PACKETSIZE}           & \code{int} \\
\code{CS_SEC_CREDTIMEOUT}      & \code{int} \\
\code{CS_SEC_SESSTIMEOUT}      & \code{int} \\

\code{CS_APPNAME}              & \code{string} \\
\code{CS_HOSTNAME}             & \code{string} \\
\code{CS_PASSWORD}             & \code{string} \\
\code{CS_SERVERNAME}           & \code{string} \\
\code{CS_USERNAME}             & \code{string} \\
\code{CS_TDS_VERSION}          & \code{string} \\
\code{CS_DS_DITBASE}           & \code{string} \\
\code{CS_DS_PASSWORD}          & \code{string} \\
\code{CS_DS_PRINCIPAL}         & \code{string} \\
\code{CS_DS_PROVIDER}          & \code{string} \\
\code{CS_SEC_KEYTAB}           & \code{string} \\
\code{CS_SEC_MECHANISM}        & \code{string} \\
\code{CS_SEC_SERVERPRINCIPAL}  & \code{string} \\
\code{CS_TRANSACTION_NAME}     & \code{string} \\

\code{CS_LOC_PROP}             & \code{CS_LOCALE} \\

\code{CS_EED_CMD}              & \code{CS_COMMAND} \\
\end{longtable}

For an explanation of the property values and get/set/clear semantics
please refer to the Sybase documentation.

The following will allocate a connection from a library context,
initialise the connection for in-line message handling, and connect to
the named server using the specified username and password.

\begin{verbatim}
def connect_db(ctx, server, user, passwd):
    status, conn = ctx.ct_con_alloc()
    if status != CS_SUCCEED:
        raise CSError(ctx, 'ct_con_alloc')
    if conn.ct_diag(CS_INIT) != CS_SUCCEED:
        raise CTError(conn, 'ct_diag')
    if conn.ct_con_props(CS_SET, CS_USERNAME, user) != CS_SUCCEED:
        raise CTError(conn, 'ct_con_props CS_USERNAME')
    if conn.ct_con_props(CS_SET, CS_PASSWORD, passwd) != CS_SUCCEED:
        raise CTError(conn, 'ct_con_props CS_PASSWORD')
    if conn.ct_connect(server) != CS_SUCCEED:
        raise CTError(conn, 'ct_connect')
    return conn
\end{verbatim}
\end{methoddesc}

\begin{methoddesc}[CS_CONNECTION]{ct_options}{action, option \optional{, value}}
Sets, retrieves and clears server query processing options for
connection.

When \var{action} is \code{CS_SET} a compatible \var{value} argument
must be supplied and the method returns the Sybase result code.  The
Sybase-CT \function{ct_options()} function is called like this:

\begin{verbatim}
/* bool option value */
status = ct_options(conn, CS_SET, option, &bool_value, CS_UNUSED, NULL);

/* int option value */
status = ct_options(conn, CS_SET, option, &int_value, CS_UNUSED, NULL);

/* string option value */
status = ct_options(conn, CS_SET, option, str_value, CS_NULLTERM, NULL);

/* locale option value */
status = ct_options(conn, CS_SET, option, locale, CS_UNUSED, NULL);
\end{verbatim}

When \var{action} is \code{CS_GET} the method returns a tuple
containing the Sybase result code and the option value.  The
Sybase-CT \function{ct_options()} function is called like this:

\begin{verbatim}
/* bool option value */
status = ct_options(conn, CS_GET, option, &bool_value, CS_UNUSED, NULL);

/* int option value */
status = ct_options(conn, CS_GET, option, &int_value, CS_UNUSED, NULL);

/* string option value */
status = ct_options(conn, CS_GET, option, str_buff, sizeof(str_buff), &buff_len);
\end{verbatim}

When \var{action} is \code{CS_CLEAR} the method returns the Sybase
result code.  The Sybase-CT \function{ct_options()} function is called
like this:

\begin{verbatim}
status = ct_options(conn, CS_CLEAR, option, NULL, CS_UNUSED, NULL);
\end{verbatim}

The recognised options are:

\begin{longtable}{l|l}
\var{option} & type \\
\hline
\code{CS_OPT_ANSINULL}       & \code{bool} \\
\code{CS_OPT_ANSIPERM}       & \code{bool} \\
\code{CS_OPT_ARITHABORT}     & \code{bool} \\
\code{CS_OPT_ARITHIGNORE}    & \code{bool} \\
\code{CS_OPT_CHAINXACTS}     & \code{bool} \\
\code{CS_OPT_CURCLOSEONXACT} & \code{bool} \\
\code{CS_OPT_FIPSFLAG}       & \code{bool} \\
\code{CS_OPT_FORCEPLAN}      & \code{bool} \\
\code{CS_OPT_FORMATONLY}     & \code{bool} \\
\code{CS_OPT_GETDATA}        & \code{bool} \\
\code{CS_OPT_NOCOUNT}        & \code{bool} \\
\code{CS_OPT_NOEXEC}         & \code{bool} \\
\code{CS_OPT_PARSEONLY}      & \code{bool} \\
\code{CS_OPT_QUOTED_IDENT}   & \code{bool} \\
\code{CS_OPT_RESTREES}       & \code{bool} \\
\code{CS_OPT_SHOWPLAN}       & \code{bool} \\
\code{CS_OPT_STATS_IO}       & \code{bool} \\
\code{CS_OPT_STATS_TIME}     & \code{bool} \\
\code{CS_OPT_STR_RTRUNC}     & \code{bool} \\
\code{CS_OPT_TRUNCIGNORE}    & \code{bool} \\

\code{CS_OPT_DATEFIRST}      & \code{int} \\
\code{CS_OPT_DATEFORMAT}     & \code{int} \\
\code{CS_OPT_ISOLATION}      & \code{int} \\
\code{CS_OPT_ROWCOUNT}       & \code{int} \\
\code{CS_OPT_TEXTSIZE}       & \code{int} \\

\code{CS_OPT_AUTHOFF}        & \code{string} \\
\code{CS_OPT_AUTHON}         & \code{string} \\
\code{CS_OPT_CURREAD}        & \code{string} \\
\code{CS_OPT_CURWRITE}       & \code{string} \\
\code{CS_OPT_IDENTITYOFF}    & \code{string} \\
\code{CS_OPT_IDENTITYON}     & \code{string} \\
\end{longtable}

For an explanation of the option values and get/set/clear semantics
please refer to the Sybase documentation.
\end{methoddesc}
